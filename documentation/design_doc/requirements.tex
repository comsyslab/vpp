\chapter{Considerations and requirements}

\section{Purpose}

The purpose of the Virtual Power Plant (VPP) platform is to:
\begin{itemize}
	\item receive measurements from sensors deployed in a building.
	\item receive current and forecast information from external sources on grid load, electricity price and more.
	\item process measurements and external information to make decisions on power usage.
	\item actuate devices deployed in a building to implement above mentioned decisions.
\end{itemize} 
The above functionality is already implemented in a number of individual applications and scripts. This system should provide the same functionality in an integrated application.

\newpage
\section{Users and privileges}

\subsection{User categories}
An initial listing of envisioned users and their access to the system is shown below:

\paragraph{Residents}
\begin{itemize}
    \item{Actuate in own home}
    \item{See data from own home}
\end{itemize}

\paragraph{Janitor}
\begin{itemize}
    \item{Should be able to do anything he/she can already do before the system}
    \item{Actuate anything not in private}
    \item{Add/remove actuators}
\end{itemize}

\paragraph{Administrative staff (ie. housing association office staff)}
\begin{itemize}
    \item{See data on some level of aggregation. Maybe just reports?}
\end{itemize}

\paragraph{Aggregator}
\begin{itemize}
    \item{multiple buildings}
    \item{specific read/actuate permissions, granted by janitor/admin. staff}
\end{itemize}

\paragraph{System administrator}
\begin{itemize}
    \item{full access}
\end{itemize}

\newpage
\subsection{Privileges requirements} \label{subsection:PrivilegesRequirements}
With outset in the above user categories and tasks, the following distinct privileges have been identified:

\paragraph{Global system privileges}
\begin{itemize}
    \item{Access to reports}
    \item{User access}
    \item{System admin access}
\end{itemize}

\paragraph{Building-specific privileges}
\begin{itemize}
    \item{Add new devices}
    \item{Remove devices}
\end{itemize}

\paragraph{Device-specific privileges}
\begin{itemize}
    \item{Read device status and data (measurements/actions)}
    \item{Configure device (configure parameters, disable)}
    \item{Remove device and data}
    \item{Actuate controller}
\end{itemize}

\newpage
\section{Use cases}

\subsection*{UC1a: Grundfoss sensor data from RabbitMQ (Priority 1)}\label{uc1}
\noindent Trigger: RabbitMQ announces new message with Grundfoss data.

\noindent Post condition: Measurements are stored in DB

\noindent Steps: 
\begin{enumerate}
    \item Message is sent from RabbitMQ to VPP \texttt{DataProvider}
    \item For each measurement in message:
        \subitem If sensor is unknown, create it and store in DB
        \subitem Store measurement (with link to sensor)
\end{enumerate}

\subsection*{UC1b: Develco sensor data from RabbitMQ (Priority 1)}\label{uc1}
\noindent Trigger: RabbitMQ announces new message with Develco data.

\noindent Post condition: Measurements are stored in DB

\subsection*{UC2: Sensor and prediction data from FTP (Priority 2)}
\noindent Trigger: Periodic check connects to configured FTP

\noindent Post condition: Measurements and predictions are stored in DB, avoiding duplication of previously posted data

\noindent Steps: 
\begin{enumerate}
    \item VPP DataProvider (with FTP adapter) connects to FTP with configured credentials
    \item The targeted file is downloaded
    \item Relevant (new) data from file is stored as measurements and predictions, as appropriate.
\end{enumerate}

\subsection*{UC3: Scheduled ControlAction is executed (Priority 3)}
\noindent Trigger: Periodic check discovers scheduled action on ventilation system

\noindent Post condition: Ventilation system setting is changed via HTTP, and the VPP DB updated to reflect action has been carried out.

\subsection*{UC4: HTTP control of Develco devices (smart plugs) (Priority 10)}
Extension of SmartAmm server with HTTP interface.

\subsection*{UC5: VPP control strategy schedules action (Priority 5)}
\noindent Trigger: Control strategy periodic check (or possibly arrival of relevant measurement/prediction data)

\noindent Post condition: New control action is stored DB

\noindent Steps: 
\begin{enumerate}
    \item Read relevant data from DB
    \item Control algorithm computes suitable action
    \item Store new action
    \item If action should be performed immediately: Notify ControllerManager
\end{enumerate}

\subsection*{UC6: Data visualization on website (Priority 5)}
\noindent Trigger: Super user wishes to view data form specific sensor

\noindent Post condition: Graph of data from sensor is displayed in browser

\noindent Steps: 
\begin{enumerate}
    \item User authenticates with username and password
    \item User specifies sensor and time range
    \item User is presented with data plot
\end{enumerate}

\subsection*{UC7: Periodic status report (Priority 6)}
\noindent Trigger: Status report is generated every 24 hours.

\noindent Post condition: Administrator receives email with report

\noindent Data in report: 
\begin{itemize}
    \item New sensors
    \item Sensors gone offline (no reports for 24 hours)
    \item Statistics on measurement data
        \subitem size (kilobytes)
        \subitem number of measurements
    \item Status on subsystems 
        \subitem uptime
\end{itemize}
Possibly use Nagios for system info


\subsection*{UC8: alarm report (Priority 7)}
\noindent Trigger: RabbitMQ has not sent data for 1 hour

\noindent Post condition: Administrator receives email with report

Configure alert subscriber

\section{Measurement data characteristics}
Grundfoss data: 2-3GBs per day. A few thousand sensors report every 5 seconds, giving approx. 50 mio. rows per day. That is 500 rows per second!

Develco: 15MB per day, 0.5 mio. rows
