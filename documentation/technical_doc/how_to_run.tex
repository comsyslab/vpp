% !TEX root =  main.tex



\chapter{How to run} \label{ch:how_to_run}

\section{Installation}

\begin{itemize}
	\item Install python 2.7.11
	\item Setup postgresql DB. Create VPP DB with suitable user access
	\item Clone from \url{https://github.com/comsyslab/vpp.git}


	\item Windows:
	\begin{itemize}
		\item Install \url{http://www.microsoft.com/en-us/download/details.aspx?id=44266} (required for pip install)
		\item Install pip
		\item Install Psycopg2 manually for windows with exe from \url{https://github.com/nwcell/psycopg2-windows}
		\item Install win32con from \url{http://sourceforge.net/projects/pywin32/files/pywin32/Build%20220/}
	\end{itemize}

	\item Linux: \texttt{pip install -r vpp\_server/resources/requirements\_psycopg2.txt}


	\item \texttt{pip install -r vpp\_server/resources/requirements.txt}
\end{itemize}


\section{Scripts}
The various scripts are placed in \texttt{vpp\_server/run} and explained below. All scripts exist in a Linux (\texttt{.sh}) and Windows (\texttt{.bat}) version.

\paragraph{\texttt{clear\_db.sh}}
Use with care. Connects to the database specified in the \texttt{[DB]} section of \texttt{config.ini} and drops and recreates the entire database schema \emph{without prompting for confirmation}. This can be used to initialize a new database.

\paragraph{\texttt{clear\_db\_dw.sh}}
Use with care. Identical to \texttt{clear\_db.sh} above, but connects to the database specified in the \texttt{[DB-DW]} section of \texttt{config.ini}. Note that this may contain data accumulated over a long time.


\paragraph{\texttt{run\_tests.sh}}
Executes all unit tests and outputs results to console.

\paragraph{\texttt{start\_server.sh}}
Launches the VPP server. \\
Logging is output in \texttt{vpp\_server/logs/console.log}.


\paragraph{\texttt{stop\_server.sh}}
Stops a running server. 
This is implemented in a very primitive fashion: The script creates a file named \texttt{stop} which the running server scans for every 5 seconds. 